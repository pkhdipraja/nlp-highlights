\documentclass[12pt]{extarticle}
\usepackage[utf8]{inputenc}
\usepackage{cite}
\usepackage{setspace}
\usepackage{amsmath}
\usepackage{indentfirst}
\setlength{\parskip}{10pt}
\title{Sequence to Sequence Modelling}
\author{Patrick Kahardipraja}
\date{September 2019}
\onehalfspacing

\begin{document}
% Make essay with style of literature review and blogposts

\maketitle
% Notes to remember: explain RNN and LSTM a bit?
% Problem of statistical machine translation?
% Applying seq2seq to many languages?
% Find seq2seq literature review
% Explain distributional representation?
% Tell about Google NMT?
% encoder - decoder
% transformer
% FOCUS ON DEVELOPMENT OF SEQ2SEQ OVER TIME AND HOW IT CHANGES
% Recent Advances: Facebook - Phrase-Based & Neural Unsupervised Machine Translation -> discuss that smt is not dead

Mapping of a sequence to another sequence is an important paradigm because of the vast amount of problems that can be formulated in this manner. For instance, in automatic speech recognition (ASR), chunks of speech signals can be mapped to sequence of phonemes while in machine translation, a sequence of words in one language can be mapped to another language. Interestingly, many other tasks such as text summarization, question answering, and image caption generation can be phrased as a sequence to sequence problem. In this paper, I will attempt to distill how sequence to sequence learning works and the motivation behind it, with a particular focus on machine translation.


\section*{Introduction}

%Brief about SMT..(Prior to NMT)NN dead in 1990s. Resurgence of Deep Neural Networks (DNNs) in the last few years have been attributed to gpu power and availability of good dataset (ImageNet). Equipped with the capability to learn features automatically with multiple levels of representation, DNNs are already proven to achieve excellent performance on task such as object recognition (AlexNet). However, although powerful.. 

Prior to neural machine translation (NMT), phrase-based stastistical machine translation (SMT) systems are widely used as it offers reliable performance. Despite its success, most of them are extremely complex and require a huge amount of effort, as it is often tailored to a specific language pair and do not generalize well to another languages. Furthermore, a lot of feature engineering are required in order to capture a specific language phenomena, which prompt researchers to explore another approach.

The resurgence of deep neural networks (DNNs) in early 2010s, thanks to faster, parallel computation using GPUs and availability of large and high-quality datasets, bring a new wave of enthusiasm in deep models. With the capability to learn features automatically with multiple, hierarchical representation, DNNs achieve excellent performance on difficult tasks in computer vision [AlexNet] and speech recognition []. Albeit powerful, DNNs has its own limitation, as it requires input and output vectors with a fixed dimension and thus not suitable for sequence to sequence problem whose lengths are unknown beforehand. In addition, DNNs also do not generalize well across temporal patterns, because each neuron has its own specific connection and as a result, a single pattern may look totally different at different timesteps.

The natural remedy for this problem is to look onto recurrent neural networks (RNNs), as it allows operations over sequences of vectors. However, mapping using RNNs typically have one-to-one correspondence between the input vectors and the output vectors. It also has another problem, as the input and output sequences can have different lengths and non-monotonic alignments. Standard RNN architecture is also not reliable for learning long-range dependencies due to the vanishing gradient problem. This issue is addressed by Sutskever et al. [Seq2Seq], where they introduce a novel and straightforward method to solve general sequence to sequence mapping using Long-Short Term Memory (LSTM) architecture. With the success of sequence to sequence learning in machine translation tasks, research in neural machine translation continue to thrive, eventually resulting in many significant improvements such as attention mechanism [Bahdanau] and subword units to deal with rare words [WordPiece]. But, before delving in too deep, I will give some brief insight into the mechanism behind RNN and LSTM in the next section. 



 

\section*{Recurrent Neural Networks} % Discuss variants of RNN, LSTM, LSTM highway, GRU?

Recurrent neural networks [Rumelhart] are type of neural network that is able to process arbitrary sequential input via combination of its internal state and input vector. At every timestep $t$, the hidden state vector $h_{t}$ is overwritten as a function of the hidden state at the previous timestep $h_{t-1}$ and the current input vector $x_{t}$. The input vector $x_{t}$ itself could be a representation of $t$-th word in a sentence, which is usually obtained using pre-trained word embeddings [GloVe, Word2Vec, ElMo]. The hidden state of RNNs can be perceived as a memory with a fixed dimensionality that can be tuned, containing distributed representation of the processed input sequence up to time $t$.

In a RNN, the forward step function consists of an affine transformation followed by a non-linear activation function. The hidden state then can be used to make predictions:
\begin{align*}
h_{t} &= a(W^{(x)}x_{t} + W^{(h)}h_{t-1} + b_{h})\\
y_{t} &= g(W^{(y)}h_{t} + b_{y}) 
\end{align*}
where in a typical application, $a$ is the hyperbolic tangent function and $g$ is the softmax function.

Although proven to be effective, RNN still has its own shortcoming. The main problem with RNN is that during training, magnitude of gradient can get weaker or stronger exponentially when backpropagating the error through time, especially with long sequences [Hochreiter, Bengio]. This phenomena is called vanishing or exploding gradient problem, which causes RNN model to experience difficulty when handling "long-term dependencies" that occur in a sequence.

Long Short Term Memory (LSTM) architecture [Hochreiter and Schmidhuber] addresses the problem of "long-term dependencies" by integrating a memory cell that is capable to memorize state that span over long sequences of time. The memory cell is controlled by gates, which have the ability to regulate how much information are added or removed in the memory cell. This means that while in a RNN a completely new hidden state is computed at every new timestep, in LSTM the hidden state is not competely overwritten, and updated according to the memory cell. The architecture of both RNNs and LSTMs are depicted in figure...  

A LSTM unit consists of 3 gates (input gate $i_{t}$, forget gate $f_{t}$, output gate $o_{t}$), memory cell $C_{t}$ and hidden state $h_{t}$. In a high-level sense, the input gate decides how much and which values will be updated, the forget gate controls the amount of information to be forgotten in the previous memory cell, and the output gate decides the hidden state by filtering the internal memory cell for each timestep. Each gate produces vectors, where their values are between 0 (completely closed) and 1 (completely open) using the sigmoid activation function. 

The formula of LSTM is described with the following equations:
\begin{align*}
i_{t} &= \sigma(W^{(i)}x_{t} + U^{(i)}h_{t-1} + b_{i}) \\
f_{t} &= \sigma(W^{(f)}x_{t} + U^{(f)}h_{t-1} + b_{f}) \\
o_{t} &= \sigma(W^{(o)}x_{t} + U^{(o)}h_{t-1} + b_{o}) \\
\tilde{C}_{t} &= \textnormal{tanh}(W^{(c)}x_{t} + U^{(c)}h_{t-1} + b_{c}) \\
C_{t} &= f_{t} \odot C_{t-1} + i_{t} \odot \tilde{C}_{t} \\
h_{t} &= o_{t} \odot \textnormal{tanh}(C_{t})
\end{align*}
where $x_{t}$ is the input vector for timestep $t$, $\sigma$ is the sigmoid activation function and $\odot$ denotes the Hadamard product of matrices. 

\subsection*{Variants}
%GRU, LSTM with peephole
Beside the standard LSTM architecture in literature that is introduced above, there also exist several popular variants which are commonly used. One of the variations which is introduced by [Gers, Schmidhueber] use "peephole connections". These connections allow the gates to look into the memory cell state in order to learn precise and stable timings. 

Other notable LSTM variants is the Gated Recurrent Unit (GRU), introduced by [Cho et al.]



\section*{Seq2Seq Model}
The first work [Cho et al.]

\section*{Recent Advances} % & related work. also include problems with Seq2Seq?

\section*{Conclusion}
\end{document}